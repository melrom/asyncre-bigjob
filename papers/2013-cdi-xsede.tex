\documentclass{sig-alternate}

\usepackage{url} 
\usepackage{subfigure} 
\usepackage{color}
\usepackage{graphicx}
\usepackage{xspace}


\usepackage[font=footnotesize]{subfig}
\DeclareCaptionType{copyrightbox}
\hyphenation{op-tical net-works semi-conduc-tor}
\newif\ifdraft
%\drafttrue
\ifdraft
\newcommand{\tbnote}[1]{ {\textcolor{magenta} { ***TcB: #1 }}}
\newcommand{\jhanote}[1]{ {\textcolor{red} { ***SJ: #1 }}}
\newcommand{\note}[1]{ {\textcolor{blue} { ***NOTE: #1 }}}
\newcommand{\abhi}[1]{ {\textcolor{green} { ***Abhinav: #1 }}}
\newcommand{\hfnote}[1]{ {\textcolor{cyan} { ***HF: #1 }}}
\newcommand{\rajib}[1]{ {\textcolor{blue} { ***RM: #1 }}}
\else
\newcommand{\rajib}[1]{}
\newcommand{\jhanote}[1]{}
%\newcommand{\note}[1]{ {\textcolor{blue} { ***NOTE: #1 }}}
\newcommand{\note}[1]{ {}}
\newcommand{\abhi}[1]{ {}}
\newcommand{\hfnote}[1]{}
\newcommand{\tbnote}[1]{ {}}
\fi

\newcommand{\pilotjob}{Pilot-Job\xspace}
\newcommand{\pilotjobs}{Pilot-Jobs\xspace}
\newcommand{\pilotdata}{Pilot-Data\xspace}
\newcommand{\pilotapi}{Pilot-API\xspace}

\begin{document}

\conferenceinfo{XSEDE13}{'13 }
 \CopyrightYear{2013}

\title{}

\numberofauthors{5} 
\author{
\alignauthor
A \\
       \affaddr{x}\\
       \affaddr{x}\\
       \email{x}
% 2nd. author
\and  % use '\and' if you need 'another row' of author names
% 3rd. author
\alignauthor 
B \\
       \affaddr{xxxy}\\
       \affaddr{xxx}\\
       \email{xxx}
% 5th. author
\alignauthor 
C \\
       \affaddr{Rutgers University}\\
       \affaddr{Piscataway, NJ 08854}\\
       \email{xxx}
}

\maketitle

\begin{abstract}

A description of the computational workflow, viz., the different
machines/resources used, how many ensembles we simulated, data
volumnes managed etc., (ii) any computational performance issues
including (a) measuring "efficiency" as the number of distributed
resources utilized goes as measured by $T_c$ (time to completion), (b)
$T_c$ as a function of the number of replicas, and (iii) a description
of the software infrastructure that is employed to enable distributed
replica-exchange on XSEDE. 

\end{abstract}

\category{H.4}{Information Systems Applications}{Miscellaneous}
%A category including the fourth, optional field follows...
\category{D.2.8}{Software Engineering}{Metrics}[complexity measures, performance measures]

\terms{Experience, Technology}

\keywords{HPC, Distributed Computing, NAMD, MD, Large Scale, XSEDE
  resources}

\section{Introduction}

\section{Scientific Problem:  Computational Requirements}
\label{sec:requirements}



\section{Describing the Replica-Exchange ``Workflow'' } \label{}


\section{Software Environment}

\subsection{BigJob}

\subsection{Async Replica-Exchange}

\section{Results on XSEDE }\label{sec:results}

\subsection{Systems Investigated} 

(i) Systems 1 -3 , (ii) $N_R$ vary from 4, 16, 32, 64, 128, 512, (iii)
$T_c = T_W + T_Q$ where terms are total time to completion, run time,
time waiting in queue.  (iv) each exchange takes place approximately
1ps (v) each replica runs for 1ns


\subsection{Computational Configuration}
(i) Fixed num. of cores [Fixed Pilots, varying resources] (ii) ``Flood"
XSEDE -- minimize the Time to Wait

\section{Discussion}


Productivity - Scientific efficiency vs. Computing efficiency!


\subsection{Future Work: Different Exchange Modes}

\subsubsection{Local}
\subsubsection{Logically Distributed}
\subsubsection{Physically Distributed}

\subsection{Other}

\subsubsection{Issue of Resiliency/Redundacy} Emilio has interesting
antidotes -- nodes crashing and RE starts limping but as nodes got
rebooted, started speeding up.  Experience --> Old to New Aspects
\texttt{get\_state()} - State monitoring capability


\section*{Acknowledgement}
\footnotesize{Additional important
  funding has been provided by NSF-ExTENCI (OCI-1007115) Important
  funding for SAGA has been provided by the UK EPSRC grant number
  GR/D0766171/1 (via OMII-UK) and HPCOPS NSF-OCI 071087. Bishop were
  supported by NIH-R01GM076356.}

\bibliographystyle{abbrv}
%\bibliography{saga,saga-old,literature,replica-exchange}
%\bibliography{saga2,literature,replica-exchange}

\end{document}
